%% This is file `IPPbeamer.tex' version 2.2 (2022/07/23),
%% it is part of the IPP-bundle
%% Beamer-slides and tcolorbox-posters for IPP
%% ----------------------------------------------------------------------------
%%
%%  Copyright (C) 2020–2023 by Marei Peischl <marei@peitex.de>
%%
%% ============================================================================
%% This work may be distributed and/or modified under the
%% conditions of the LaTeX Project Public License, either version 1.3c
%% of this license or (at your option) any later version.
%% The latest version of this license is in
%% http://www.latex-project.org/lppl.txt
%% and version 1.3c or later is part of all distributions of LaTeX
%% version 2008/05/04 or later.
%%
%% This work has the LPPL maintenance status `maintained'.
%%
%% Thecurrent maintainer of this work is
%%   Marei Peischl <marei@peitex.de>
%%
%% ============================================================================
%%
\documentclass[
	english,% main language as global option
	logo=false,% logo to be used in the headline. Pre-defined values are: asdex, w7x, false. Initial value is empty
	eurofusion=false, %initial value is false
	titlegraphic=true, %initial value is false
	]{ippbeamer}

\usepackage{amsmath}
\usepackage{amsfonts}
\usepackage{amssymb}
\usepackage{amsthm}
\usepackage{bm}
\usepackage{caption}
\usepackage{xcolor}
\usepackage{csquotes}
\usepackage{enumerate}
\usepackage{faktor}
\usepackage{graphicx}
\usepackage{hyperref}
\usepackage{listings}
\usepackage{mathrsfs}
\usepackage{mathtools}
\usepackage{pgf,tikz}
\usepackage{pgfplots, pgfplotstable}
% \usepackage{showlabels}
\usepackage{svg}
\usepackage{tikz-cd}
\usepackage{url}

\pgfplotsset{compat=1.18}
\usetikzlibrary{arrows}
\usetikzlibrary{decorations.pathreplacing,decorations.markings}

\definecolor{codegreen}{rgb}{0,0.6,0}
\definecolor{codegray}{rgb}{0.5,0.5,0.5}
\definecolor{codepurple}{rgb}{0.58,0,0.82}
\definecolor{backcolour}{rgb}{0.95,0.95,0.92}


\lstdefinestyle{mystyle}{
    backgroundcolor=\color{backcolour},   
    commentstyle=\color{codegreen},
    keywordstyle=\color{magenta},
    numberstyle=\tiny\color{codegray},
    stringstyle=\color{codepurple},
    basicstyle=\ttfamily\footnotesize,
    breakatwhitespace=false,         
    breaklines=true,                 
    captionpos=b,                    
    keepspaces=true,                 
    % numbers=left,                    
    numbersep=5pt,                  
    showspaces=false,                
    showstringspaces=false,
    showtabs=false,                  
    tabsize=2
}

\lstset{style=mystyle}

\captionsetup[figure]{labelformat=empty}

% \newtheorem{lemma}{Lemma}[section]
% \numberwithin{lemma}{section}
% \newtheorem{corollary}[lemma]{Corollary}
% \newtheorem{proposition}[lemma]{Proposition}
% \newtheorem{theorem}[lemma]{Theorem}

% \theoremstyle{definition}
% \newtheorem{assumption}[lemma]{Assumption}
% \newtheorem{definition}[lemma]{Definition}
% \newtheorem{example}[lemma]{Example}
% \newtheorem{remark}[lemma]{Remark}
% \newtheorem{problem}[lemma]{Problem}

\DeclareMathOperator{\grad}{grad}
\DeclareMathOperator{\curl}{curl}
\DeclareMathOperator{\diver}{div}

\newtheorem{assumption}{Assumption}




% some adjustments which are generally not available
\let\code\texttt
\let\file\texttt

% extended graphicspath to use separate image directories without the bundle being installed
\graphicspath{{}{img/}{img/titlegraphic/}}

\title{Multipatch ????}
% \subtitle{presentations using \LaTeX}
\author{Pauline Vidal, Alexander Hoffmann}
% \institute{\inst{1}pei\TeX{} \and\inst{2}Institut 2}
\date{\today}
% Additional Logo to be placed in the footer
% \logo{\includegraphics[height=\height]{example-image}}

\begin{document}


\frame{\titlepage}
\begin{frame}{Solving Poisson with CONGA}
	We want to solve the Poisson equation
	\begin{align*}
		-\diver (\nu \grad \phi) = \rho
	\end{align*}
	using a the CONGA approach on a 2D multipatch domain. 
	\begin{itemize}
		\item Have finite element space $V_h$ with jump discontinuities across edges
		\item Subspace $V_h^c \subseteq V_h$ with functions conforming to some global
				regularity constraint
		\item Define projection $P_h: V_h \rightarrow V_h^c$
				and discrete differential operator $\grad_h \vcentcolon= \grad P_h$
		\item Discretize Poisson equation weakly using these operators
	\end{itemize}
	\begin{alertblock}{}
		\begin{itemize}
			\item Already implemented in Psydac and successfully applied to several problems. 
			\item Probably easier to generalize to complicated geometries than other approaches
					like using different splines e.g.
		\end{itemize}
	\end{alertblock}
\end{frame}

\begin{frame}{Patch data -- Exchange}
	\begin{itemize}
		\item Mesh-points,
		\item Local sums to compute the derivatives at the interfaces
				\begin{align*}
					\sum_{x_i \in \text{global space}} \alpha_i s(x_i) = \sum_{p \in\text{Patches}} 
							\sum_{x_i \in p} \alpha_i s(x_i),
				\end{align*}
		\item Characteristic feet outside of the patch,
		\item Interpolated values for $\mathbf{A}$ and $\rho$.
	\end{itemize}
\end{frame}
\begin{frame}{Patch data -- Storage}
	\begin{itemize}
		\item Mesh points, dimension (\texttt{DimXi}, \texttt{DimYi}), mapping,
				\texttt{SplineBuilder}, metadata, 
		\item Boundary condition of global domain if an edge of the patch is on the global boundary,
		\item Values of functions $\rho$, $\phi$, $\mathbf{A}$ on mesh points,
		\item Spline coefficients of functions ($\rho$, $\phi$, $\mathbf{A}$),
		\item Reference to global domain class.
	\end{itemize}
\end{frame}

\begin{frame}{Global domain}
	\begin{itemize}
		\item Global domain class
		\begin{itemize}
			\item References to patches,
			\item 'Connectivity' class which encodes the geometrical information.
		\end{itemize}
		\item Connectivity class
		\begin{itemize}
			\item Identify edges and corners of different patches (do we need to identify
					corners of same patch e.g. when it closes on itself?)
			\item For T-joint, identify sections of edges with sections of other edges, 
					place corners in the middle of edges.
		\end{itemize}
		
	\end{itemize}
	\end{frame}





\end{document}

