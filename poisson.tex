\documentclass[presentation.tex]{subfiles}

\begin{document}

\section{Poisson problem on multipatch}
We want to solve the Poisson equation
\begin{align*}
    -\diver (\nu \grad \phi) = \rho
\end{align*}
using a the CONGA approach on a multipatch domain. 

The CONGA approach has the main advantage that it uses patch-local 
degrees of freedom and basis functions. The coupling of adjacent patches 
happens through the use of projection operators onto conforming subspaces. 
This approach has been studied by Martin Campos Pinto and Yaman Güçlü in the context of FEEC. 
The projection operators are implemented in Psydac and 
have been successfully applied to different problems in electromagnetics. 


\subsection*{Very short introduction to the CONGA approach}
This short section is just a very short
introduction of the ideas in our context. 
For more details see \href{https://arxiv.org/abs/2109.02553}{Campos Pinto, Güçlü (2024) - Broken-FEEC discretizations and Hodge Laplace 
problems} and \href{https://link.springer.com/article/10.1007/s10915-023-02351-x}{Güçlü, 
            Hadjout, Campos Pinto (2023) - A Broken FEEC Framework for Electromagnetic Problems on Mapped Multipatch Domains}


We define finite element spaces $V_h$ and $V_h^c$. 
We assume that our domain $\Omega$ is given as a disjoint union of patches 
$\Omega_k$, $k=1, ..., K$.
Let $V_h(\Omega_k)$ be the finite element space on $\Omega_k$ for every $k$. 
Define the basis of $V_h$ by taking the union of all basis functions of $V_h(\Omega_k)$ for all $k$ and 
extending them by zero outside of their respective patch $\Omega_k$. 
We recognize that the functions in $V_h$ can have jumps across 
patch boundaries and are thus not continuous. 
Hence, we call $V_h$ \textit{broken}.
Let $V$ be a subspace of $L^2(\Omega)$ with functions fulfilling some regularity
requirement e.g. $V=H^1(\Omega)$.

We further define the conforming subspace
\begin{align*}
    V_h^c \vcentcolon= V_h \cap V
\end{align*}
i.e. the functions in $V_h$ with appropriate regularity. We then define
projection operators
\begin{align*}
    P_h: V_h \rightarrow V_h^c.
\end{align*}
and the discrete differential operator
\begin{align*}
    \grad_h \vcentcolon= \grad \, P_h.
\end{align*}
For an operator $A$ defined on a subspace of $L^2(\Omega)$ we denote the 
adjoint w.r.t. the $L^2$ inner product as $A^*$, if it exists.
Then we define the CONGA Laplacian operator as $\grad_h^* (\nu \grad_h)$
and introduce a stabilized version via
\begin{align*}
    \grad_h^*(\nu \grad_h) + \alpha (I-P_h^*)(I-P_h)
\end{align*}
where $\alpha > 0$ is some constant and $I$ is the identity. This is discretized weakly by discretizing the bilinear
form $a_h: V_h \times V_h \rightarrow \mathbb{R}$
\begin{align*}
    a_h(\phi_h, \psi_h) = \int_\Omega \nu \grad_h \phi_h \cdot \grad_h \psi_h \, dx 
                        + \int_\Omega (I-P_h)\phi_h \, (I-P_h)\psi_h \, dx
\end{align*}

\subsection*{Applying the CONGA approach}
\begin{itemize}
    \item \textbf{Question:} Does the advection field $\mathbf{A}$ have to be
            continuous? Emily thinks so. Then we need to enforce global $C^1$-regularity on $\phi$ 
    $\rightarrow$ Use CONGA approach with $C^1$ conforming projection $P_h$
    \item   The conforming projection is computed by taking suitable averages of spline coefficients close to 
            the boundary of the patch so it only acts on splines close to the edge
    \item   Need the mesh information from neighboring patches
            to construct the projection matrices
    \item Spline coefficients of $\rho$ needed to assemble the right hand side
    \item Poisson solver needs to access all spline coefficients from all patches
            $\rightarrow$ need global managing of the splines, mesh etc 
            to assemble everything
    \item \textbf{Question:} Do we make the assembly of the stiffness matrix completely global?
\end{itemize}





\end{document}
